\documentclass[12pt]{article}
  \usepackage{geometry}
  \geometry{
    a4paper,
    total={170mm,257mm},
    left=20mm,
    top=20mm,
  }
%Packages
\usepackage{polski}
\usepackage[T1]{fontenc}
\usepackage[utf8]{inputenc}
\usepackage{color}   %May be necessary if you want to color links
\usepackage{listings}
\usepackage{graphicx}
\usepackage{float}
\usepackage[hidelinks,linktoc=all]{hyperref}
\usepackage{hyperref}

%listings config
\definecolor{mygreen}{rgb}{0,0.6,0}
\definecolor{mygray}{rgb}{0.5,0.5,0.5}
\definecolor{mymauve}{rgb}{0.58,0,0.82}

\lstset{ %
  backgroundcolor=\color{white},   % choose the background color; you must add \usepackage{color} or \usepackage{xcolor}; should come as last argument
  basicstyle=\footnotesize,        % the size of the fonts that are used for the code
  breakatwhitespace=false,         % sets if automatic breaks should only happen at whitespace
  breaklines=true,                 % sets automatic line breaking
  captionpos=b,                    % sets the caption-position to bottom
  commentstyle=\color{mygreen},    % comment style
  deletekeywords={...},            % if you want to delete keywords from the given language
  escapeinside={\%*}{*)},          % if you want to add LaTeX within your code
  extendedchars=true,              % lets you use non-ASCII characters; for 8-bits encodings only, does not work with UTF-8
  frame=single,	                   % adds a frame around the code
  keepspaces=true,                 % keeps spaces in text, useful for keeping indentation of code (possibly needs columns=flexible)
  keywordstyle=\color{blue},       % keyword style
  language=Octave,                 % the language of the code
  morekeywords={*,...},            % if you want to add more keywords to the set
  numbers=left,                    % where to put the line-numbers; possible values are (none, left, right)
  numbersep=5pt,                   % how far the line-numbers are from the code
  numberstyle=\normalsize\color{mygreen}, % the style that is used for the line-numbers
  rulecolor=\color{black},         % if not set, the frame-color may be changed on line-breaks within not-black text (e.g. comments (green here))
  showspaces=false,                % show spaces everywhere adding particular underscores; it overrides 'showstringspaces'
  showstringspaces=false,          % underline spaces within strings only
  showtabs=false,                  % show tabs within strings adding particular underscores
  stepnumber=1,                    % the step between two line-numbers. If it's 1, each line will be numbered
  stringstyle=\color{mymauve},     % string literal style
  tabsize=2	                   % sets default tabsize to 2 spaces                   % show the filename of files included with \lstinputlisting; also try caption instead of title
}
%Define more keywords
\lstdefinelanguage[Oracle]{SQL}[]{SQL}{
  morekeywords={FUNCTION, RETURNS, RETURN, BEGIN, TRY, CATCH, PROCEDURE, TRIGGER, VIEW},
}

\lstset{language=[Oracle]SQL,
       }

%Content
\begin{document}

\title{Bazy Danych}
\author{Wojciech Geisler}
\date{}
\maketitle
%\newpage

\tableofcontents

\clearpage

\section{Tabele}

\subsubsection{WYCIECZKI}
\lstinputlisting{tables/wycieczki.sql}

\subsubsection{OSOBY}
\lstinputlisting{tables/osoby.sql}

\subsubsection{REZERWACJE}
\lstinputlisting{tables/rezerwacje.sql}

\subsection{Constraints}
\lstinputlisting{constraints.sql}

\section{Przykładowe dane}
\lstinputlisting{seed.sql}

\section{Widoki}


\subsection{wycieczki\_osoby}
\lstinputlisting{views/wycieczki_osoby.sql}

\subsection{wycieczki\_osoby\_potwierdzone}
\lstinputlisting{views/wycieczki_osoby_potwierdzone.sql}

\subsection{wycieczki\_przyszle}
\lstinputlisting{views/wycieczki_przyszle.sql}

\subsection{wycieczki\_miejsca}
\lstinputlisting{views/wycieczki_miejsca.sql}

\subsection{dostepne\_wycieczki\_view}
\lstinputlisting{views/dostepne_wycieczki_view.sql}

\subsection{rezerwacje\_do\_anulowania}
\lstinputlisting{views/rezerwacje_do_anulowania.sql}

\section{Funkcje pobierające dane}

\subsection{uczestnicy\_wycieczki}
\lstinputlisting{procedures/uczestnicy_wycieczki.sql}

\subsection{rezerwacje\_osoby}
\lstinputlisting{procedures/rezerwacje_osoby.sql}

\subsection{przyszle\_rezerwacje\_osoby.sql}
\lstinputlisting{procedures/przyszle_rezerwacje_osoby.sql}

\subsection{dostepne\_wycieczki}
\lstinputlisting{procedures/dostepne_wycieczki.sql}

\section{Procedury modyfikujące dane}

Prezentowane procedury uwzględniają użycie tabeli REZERWACJE_LOG dodanej w punkcie (6).

\subsection{dodaj\_rezerwacje}
\lstinputlisting{procedures/dodaj_rezerwacje.sql}

\subsection{zmiana\_statusu\_rezerwacji}
\lstinputlisting{procedures/zmiana_statusu_rezerwacji.sql}

\subsection{zmien\_liczbe\_miejsc}
\lstinputlisting{procedures/zmien_liczbe_miejsc.sql}

\section{Dodanie tabeli dziennikującej}

\lstinputlisting{tables/rezerwacje_log.sql.sql}

\section{Dodanie pola liczba_wolnych_miejsc}

\lstinputlisting{tables/wycieczki_liczba_wolnych_miejsc.sql}

\subsection{Procedura aktualizująca}
\lstinputlisting{procedures/przelicz.sql}

\subsection{Dostosowane widoki}

\subsubsection{wycieczki\_miejsca2}
\lstinputlisting{views/wycieczki_miejsca2.sql}

\subsubsection{dostepne\_wycieczki\_view2}
\lstinputlisting{views/dostepne_wycieczki_view2.sql}

\subsection{Dostosowane procedury}

\subsubsection{dodaj\_rezerwacje2}
\lstinputlisting{procedures/dodaj_rezerwacje2.sql}

\subsubsection{dostepne\_wycieczki2}
\lstinputlisting{procedures/dostepne_wycieczki2.sql}


\subsubsection{zmiana\_statusu\_rezerwacji2}
\lstinputlisting{procedures/zmiana_statusu_rezerwacji2.sql}


\subsubsection{zmien\_liczbe\_miejsc2}
\lstinputlisting{procedures/zmien_liczbe_miejsc2.sql}

\section{Dodanie triggerów}


\subsection{triger obsługujący dodanie rezerwacji}
\lstinputlisting{triggers/nowa_rezerwacja_trigger.sql}


\subsection{triger obsługujący zmianę statusu}
\lstinputlisting{triggers/zmiana_statusu_trigger.sql}

\subsection{triger zabraniający usunięcia rezerwacji}
\lstinputlisting{triggers/usuniecie_rezerwacji_trigger.sql}

\subsection{triger obsługujący zmianę liczby miejsc na poziomie wycieczki}
\lstinputlisting{triggers/zmiana_miejsc_trigger.sql}

\subsection{Dostosowane procedury}

\subsubsection{dodaj\_rezerwacje3}
\lstinputlisting{procedures/dodaj_rezerwacje3.sql}

\subsubsection{zmiana\_statusu\_rezerwacji3}
\lstinputlisting{procedures/zmiana_statusu_rezerwacji3.sql}

\subsubsection{zmien\_liczbe\_miejsc3}
\lstinputlisting{procedures/zmien_liczbe_miejsc3.sql}

\end{document}

