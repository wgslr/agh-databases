\documentclass[12pt]{article}
  \usepackage{geometry}
  \geometry{
    a4paper,
    total={170mm,257mm},
    left=20mm,
    top=20mm,
  }
%Packages
\usepackage{polski}
\usepackage[T1]{fontenc}
\usepackage[utf8]{inputenc}
\usepackage{color}   %May be necessary if you want to color links
\usepackage{listings}
\usepackage{graphicx}
\usepackage{float}
\usepackage[hidelinks,linktoc=all]{hyperref}
\usepackage{hyperref}

%listings config
\definecolor{mygreen}{rgb}{0,0.6,0}
\definecolor{mygray}{rgb}{0.5,0.5,0.5}
\definecolor{mymauve}{rgb}{0.58,0,0.82}

\lstset{ %
  backgroundcolor=\color{white},   % choose the background color; you must add \usepackage{color} or \usepackage{xcolor}; should come as last argument
  basicstyle=\footnotesize,        % the size of the fonts that are used for the code
  breakatwhitespace=false,         % sets if automatic breaks should only happen at whitespace
  breaklines=true,                 % sets automatic line breaking
  captionpos=b,                    % sets the caption-position to bottom
  commentstyle=\color{mygreen},    % comment style
  deletekeywords={...},            % if you want to delete keywords from the given language
  escapeinside={\%*}{*)},          % if you want to add LaTeX within your code
  extendedchars=true,              % lets you use non-ASCII characters; for 8-bits encodings only, does not work with UTF-8
  frame=single,	                   % adds a frame around the code
  keepspaces=true,                 % keeps spaces in text, useful for keeping indentation of code (possibly needs columns=flexible)
  keywordstyle=\color{blue},       % keyword style
  language=Octave,                 % the language of the code
  morekeywords={*,...},            % if you want to add more keywords to the set
  numbers=left,                    % where to put the line-numbers; possible values are (none, left, right)
  numbersep=5pt,                   % how far the line-numbers are from the code
  numberstyle=\normalsize\color{mygreen}, % the style that is used for the line-numbers
  rulecolor=\color{black},         % if not set, the frame-color may be changed on line-breaks within not-black text (e.g. comments (green here))
  showspaces=false,                % show spaces everywhere adding particular underscores; it overrides 'showstringspaces'
  showstringspaces=false,          % underline spaces within strings only
  showtabs=false,                  % show tabs within strings adding particular underscores
  stepnumber=1,                    % the step between two line-numbers. If it's 1, each line will be numbered
  stringstyle=\color{mymauve},     % string literal style
  tabsize=2	                   % sets default tabsize to 2 spaces                   % show the filename of files included with \lstinputlisting; also try caption instead of title
}
%Define more keywords
\lstdefinelanguage[Transact]{SQL}[]{SQL}{
  morekeywords={FUNCTION, RETURNS, RETURN, BEGIN, TRY, CATCH, PROCEDURE, TRIGGER, VIEW},
}

\lstset{language=[Transact]SQL,
       }

%Content
\begin{document}

\title{Podstawy Baz Danych:\\System zarządzania konferencjami}
\author{Wojciech Geisler, Artur Jopek}
\date{}
\maketitle
%\newpage

\tableofcontents

\clearpage

\section{Analiza wymagań}
Tworzona baza danych przeznaczona jest dla firmy organizującej konferencji. Musi przechowywać informacje związane z przebiegiem konferencji, jej uczestnikami i ich płatnościami za udział w konferencjach.

%Można wyróżnić następujących aktorów korzystających z informacji w bazie danych:
%
%\subsection{Organizator}
%
%\subsection{Klient}
%Klienci rejestrują się na konferencje, wybierając w których dniach i warsztatach chcą uczestniczyć. 
%
%\subsubsection{Klient biznesowy}
%Firma wykupująca miejsca na konferencji ma możliwośc zarezerwowania ich w wybranej liczbie i przesłania dokładnych danych uczestników w czasie późniejszym - o maksymalnie 2 tygodnie. Uczestnicy konferencji obecni z jej ramienia powinni mieć informację o swoim pracodawcy umieszczoną na identyfikator imiennych.



Można wyróżnić następujących aktorów korzystających z informacji w bazie danych i potrzebną im funkcjonalność:

\subsection{Organizator}
\begin{enumerate}
\item Zaplanowanie nadchodzących konferencji wraz z odbywającymi się na nich wykładami i warsztatmi
\item Zaplanowanie zmian w cenie biletów na konferencje wraz ze zbliżającym się jej rozpocząciem
\item Monitorowanie płatności od klientów, możliwość łatwej identyfikacji klientów zalegających z płatnościami
\item Zapewnienie, że nieopłacone zamówienia nie będą honorowane (zostaną anulowane)
\item Zapewnienie, że dany uczestnik zarejestrowany będzie tylko na jeden warsztat jednocześnie
\item Dostęp do listy uczestników danego dnai konferencji bądź warsztatu
\item Przygotowanie identyfikatoróœ imiennych dla uczestników konferencji
\item Dostęp do statystyk informujących o najbardziej zaangażowanych klientach - analiza częstotliwości korzystania z usług firmy i płatności
\end{enumerate}

\subsection{Klient}
\begin{enumerate}
\item Dostęp do informacji o dostępnych warsztatach i wykładach na konferencji
\item Zapoznanie się z cenami wstępu na dni konferencji i warsztaty w zależności od daty złożenia zamówienia
\item Rezerwacja wstępu na dany dzień konferencji i miejsc na warsztatach
\item Przekazanie swoich danych osobowych na potrzeby organizacji konferencji
\end{enumerate}

\subsubsection{Klient biznesowy}
\begin{enumerate}
\item Zbiorowa rezerwacja miejsc dla wielu uczestników i jednorazowe ich opłacenie
\item Możliwość podania konkretnych danych osobowych w terminie późniejszym niż sama rezerwacja miejsc
\end{enumerate}

\section{Schemat bazy danych}

% TODO opisać poszczególne

\subsection{Diagram}
\begin{figure}[H]
\centering
\includegraphics[width=1\textwidth]{./schema.png}
\end{figure}


\subsection{Tabele}

\subsubsection{conference}

Opisuje konferencję organizowaną przez firmę. Informuje o jej lokalizacji (adresie) i obowiązującej zniżce studenckiej. \\

\textbf{Pola tabeli:}
\begin{description}
\item [conference\_id] Identyfikator konferencji
\item [name] Nazwa konferencji
\item [street] Ulica - część adresu miejsca odbywania się konferencji
\item [building\_number] Numer budynku - część adresu miejsca odbywania się konferencji
\item [city] Miasto - część adresu miejsca odbywania się konferencji
\item [student\_discount\_percent] Procentowo wyrażona zniżka przysługująca studentom
\end{description}

\lstinputlisting{sql/tables/conference.sql}

\subsubsection{conference\_day.sql}

Jeden z potencjalnie wielu dni konferencji. \\

\textbf{Pola tabeli:}
\begin{description}
\item [conference\_day\_id] Identyfikator dnia konferencji
\item [conference\_id] Identyfikator konferencji do której dzień przynależy
\item [date] Data tego dnia
\item [attendees\_limit] Maksymalna liczba uczestników
\end{description}

\lstinputlisting{sql/tables/conference_day.sql}

\subsubsection{lecture}

Opisuje wykład odbywający się w którymś dniu konferencji.\\

\textbf{Pola tabeli:}
\begin{description}
\item [lecture\_id] Identyfikator wykładu
\item [conference\_day\_id] Identyfikator dnia konferencji w który odbywa się wykład
\item [start] Godzina rozpoczęcia
\item [end] Godzina rozpoczęcia
\item [name] Tytuł wykładu
\item [room] Miejsce odbywania się wykładu
\end{description}
\lstinputlisting{sql/tables/lecture.sql}


\subsubsection{workshop}
Opisuje warsztat odbywający się w którymś dniu konferencji. Warszat posiada ograniczenie na ilość uczestników. \\

\textbf{Pola tabeli:}
\begin{description}
\item [workshop\_id] Identyfikator warsztatu
\item [conference\_day\_id] Identyfikator dnia konferencji w który odbywa się warszat
\item [start] Godzina rozpoczęcia
\item [end] Godzina rozpoczęcia
\item [name] Tytuł warsztatu
\item [description] Opis warsztatu
\item [price] Koszt uczestnicwa w warsztaci
\item [attendees\_limit] Maksymalna liczba uczestników
\end{description}

\lstinputlisting{sql/tables/workshop.sql}


\subsubsection{buyer}

Opisuje klienta kupującego bilety wstępu na konferencje. Może być firmą lub osobą prywatną (pole \texttt{is\_company}). Jeden klient (jeśli jest firmą) może złożyć zamówienie rezerwujące wiele miejsc na konferencji. \\

\textbf{Pola tabeli:}
\begin{description}
\item [buyer\_id] Identyfikator klienta
\item [name] Nazwa klienta
\item [billing\_address] Adres fakturowania klienta
\item [phone] Telefon kontaktowy klienta
\item [is\_company] Oznaczenie czy dany klient jest firmą
\end{description}
\lstinputlisting{sql/tables/buyer.sql}

\subsubsection{order}
Opisuje rezerwację miejsc na konferencji przez klienta.\\

\textbf{Pola tabeli:}
\begin{description}
\item [order\_id] Identyfikator zamówienia
\item [buyer\_id] Identyfikator klienta składającego zamówienie
\item [date] Data złożenia zamówienia
\item [is\_canceled] Oznaczenie anulowanego zamówienia
\end{description}
\lstinputlisting{sql/tables/order.sql}

\subsubsection{payment}
Opisuje płatnośc klienta za zamówienie.\\

\textbf{Pola tabeli:}
\begin{description}
\item [payment\_id] Identyfikator płatności
\item [order\_id] Identyfikator opłacanego zamówienia
\item [date] Data dokonania płatności
\item [means] Forma płatności. Może przyjmować jedną spośród wartości:
  \begin{itemize}
  \item cash
  \item transfer
  \item cheque
  \item card
  \end{itemize}
\end{description}
\lstinputlisting{sql/tables/payment.sql}

\subsubsection{price}
Opisuje progi cenowe konferencji. \\

\textbf{Pola tabeli:}
\begin{description}
\item [price\_id] Identyfikator progu cenowego
\item [value] Kwota ceny
\item [conference\_day\_id] Identyfikator dnia konferencji którego dotyczy próg cenowy
\item [end\_date] Ostatni dzień obowiazywania progu cenowego
\end{description}

\lstinputlisting{sql/tables/price.sql}

\subsubsection{attendee}
Opisuje dane pojedynczego uczestnika konferencji.\\

\textbf{Pola tabeli:}
\begin{description}
\item [attendee\_id] Identyfikator uczesntika konferencji
\item [name] Imię i nazwisko uczestnika
\item [email] Adres e-mail uczestnika
\end{description}
\lstinputlisting{sql/tables/attendee.sql}

\subsubsection{order\_item\_day}
Realizuje powiązanie wiele do wielu między zamówieniem a dniami konferencji na które jest składane. \\

\textbf{Pola tabeli:}
\begin{description}
\item [order\_item\_day\_\_id] Identyfikator
\item [order\_id] Identyfikator zamówienia
\item [conference\_day\_id] Identyfikator dnia konferencji
\item [attendees\_count] Liczba zarezerwowanych miejsc
\item [students\_count] Liczba studentów pośród zarezerwowanych miejsc
\item [is\_canceled] Oznaczenie anulowanego zamówienia, zawsze równe wartości w powiązanym rekordzie tabeli \texttt{order}
\end{description}

\lstinputlisting{sql/tables/order_item_day.sql}

\subsubsection{order\_item\_workshop}
Realizuje powiązanie wiele do wielu między rezerwacją miejsc w dniu konferencji a rezerwacją miejsca na konkretnym warsztacie. \\

\textbf{Pola tabeli:}
\begin{description}
\item [order\_item\_workshop\_id] Identyfikator
\item [workshop\_id] Identyfikator warsztatu
\item [order\_item\_day\_id] Identyfikator rezerwacji dnia konferencji
\item [attendees\_count] Liczba zarezerwowanych miejsc
\item [students\_count] Liczba studentów pośród zarezerwowanych miejsc
\item [is\_canceled] Oznaczenie anulowanego zamówienia, zawsze równe wartości w powiązanym rekordzie tabeli \texttt{order}
\end{description}
\lstinputlisting{sql/tables/order_item_workshop.sql}

\subsubsection{attendee\_order\_item\_day}
Realizuje powiązanie wiele do wielu między rezerwacją miejsc w dniu konferencji danymi poszczególnych uczestników. \\


\textbf{Pola tabeli:}
\begin{description}
\item [attendee\_order\_item\_day\_id] Identyfikator
\item [order\_item\_day\_id] Identyfikator rezerwacji dnia konferencji
\item [attendee\_id] Identyfikator uczestnika
\item [is\_student] Oznaczenie, czy uczestnik zapisany jest na uczelnię jako student
\item [student\_no] Numer legitymacji. Musi być podany jeśli \texttt{is\_student} jest ustawione.
\end{description}

\lstinputlisting{sql/tables/attendee_order_item_day.sql}

\subsubsection{attendee\_order\_item\_workshop}
Realizuje powiązanie wiele do wielu między rezerwacją miejsc w dniu konferencji danymi poszczególnych uczestników. \\

\textbf{Pola tabeli:}
\begin{description}
\item [attendee\_order\_item\_workshop\_id] Identyfikator
\item [order\_item\_workshop\_id] Identyfikator rezerwacji miejsca na warsztacie
\item [attendee\_order\_item\_day\_id] Identyfikator rezerwacji dnia konferencji
\end{description}

\lstinputlisting{sql/tables/attendee_order_item_workshop.sql}

\subsection{Definicje kluczy obcych}

\lstinputlisting{sql/foreign_keys.sql}

\section{Widoki}

\subsection{Nadchodzące konferencje}
Nadchodzące konferencje w rozbiciu na poszczególne dni, z informacją o obecnej cenie
(przy zakupie jednego dnia oraz podsumowanie dla zakupu wejścia na wszystkie dni) oraz o liczbie wolnych miejsc (na dany dzień i kupując na całą konferencję).
\lstinputlisting{sql/views/upcoming_conferences_view.sql}

\subsection{Identyfikatory uczestników}
Dane potrzebne do przygotowania identyfikatorów dla uczestników konferencji.
\lstinputlisting{sql/views/badges_view.sql}

\subsection{Brakujące dane uczestników}
Lista klientów którzy nie przekazali danych uczestników dla swoich rezerwacji w wymaganym terminie.
\lstinputlisting{sql/views/missing_information_view.sql}

\subsection{Statystyki klientów}
Lista klientów wraz z sumą ich płatności.
\lstinputlisting{sql/views/client_stats_view.sql}

\subsection{Statystyki miesięcznych przychodów}
Przychody firmy w rozbiciu na miesiące.
\lstinputlisting{sql/views/monthly_income_view.sql}

\subsection{Wartości zamówień}
Zamówienia wraz z ich łączną wartością.
\lstinputlisting{sql/views/orders_view.sql}

\subsection{Dostępne miejsca na warsztatach}
Tabela warsztatów rozszerzona o informacje o dostępnych jeszcze miejscach
\lstinputlisting{sql/views/workshops_view.sql}

\section{Funkcje}
\subsection{Konferencje}
%-------------------------------
% How to contribute
%-------------------------------
%\subsubsection{Nazwa}
%Opis
%\lstinputlisting{sql/functions/nazwa.sql}
%-------------------------------

\subsubsection{conference\_days}
Zwraca wpisy w tabeli conference\_days odpowiadające dniom konferencji, której id jest podane.
\lstinputlisting{sql/functions/conference_days.sql}

\subsubsection{conference\_day\_available\_places}
Zwraca ilość wolnych miejsc na dany dzień konferencji.
\lstinputlisting{sql/functions/conference_day_available_places.sql}

\subsubsection{day\_price\_on\_date}
Zwraca koszt dnia konferencji przy zamówieniu złożonym w podanym dniu
\lstinputlisting{sql/functions/day_price_on_date.sql}

\subsubsection{conference\_day\_attendees}
Zwraca tabele danych uczestnikow danej konferencji
\lstinputlisting{sql/functions/conference_day_attendees.sql}

\subsubsection{conference\_days\_with\_prices}
Zwraca tabele danych dni konferencji, z cenami i wolnymi miejscami.
\lstinputlisting{sql/functions/conference_days_with_prices.sql}

\subsection{Warsztaty}
\subsubsection{workshop\_available\_places}
Zwraca ilość wolnych miejsc na dany warsztat.
\lstinputlisting{sql/functions/workshop_available_places.sql}

\subsubsection{workshop\_list\_by\_conference}
Zwraca tabele danych warsztatów wrac z wolnymi miejscami, dla danej konferencji.
\lstinputlisting{sql/functions/workshop_list_by_conference.sql}

\subsection{Zamówienia}
\subsubsection{order\_item\_day\_price}
Zwraca cenę obowiązującą na dany dzień konferencji w dniu złożenia zamówienia.
\lstinputlisting{sql/functions/order_item_day_price.sql}

\subsubsection{workshops\_in\_order\_value}
Zwraca łączny wartość miejsc na warsztaty w obrębie danego zamówienia.
\lstinputlisting{sql/functions/order_item_day_price.sql}

\section{Procedury}

\subsection{Wyświetlające}
\subsubsection{Liczby uczestników według dnia konferencji}
\lstinputlisting{sql/procedures/get_conference_days_attendees_count.sql}

\subsubsection{Statystyki odwiedzania wielu dni tej samej konferencji}

Procedura podsumowuje ilu jest takich uczestników, którzy przyjdą na jeden, dwa itd. dni danej konferencji.

\lstinputlisting{sql/procedures/get_days_per_attendee_stat.sql}

\subsubsection{Lista wydarzeń w dniu konferencji}
Zbiorcza lista wydarzeń - warsztatów i wykładów - odbywajacych się w danym dniu konferencji.
\lstinputlisting{sql/procedures/get_events_in_a_day.sql}

\subsection{Dodające}
Przygotowanie procedur w celu wprowadzania danych do bazy pozwala na bardziej przyjazną dla użytkownika obsługę błędów.

\subsubsection{Dodawanie nowej konferencji}
\lstinputlisting{sql/procedures/add_conference.sql}

\subsubsection{Dodawanie dnia konferencji}
\lstinputlisting{sql/procedures/add_conference_day.sql}

\subsubsection{Dodawanie dnia do zamówienia}
\lstinputlisting{sql/procedures/add_order_day.sql}

\subsubsection{Dodawanie zamówienia klienta prywatnego}
\lstinputlisting{sql/procedures/add_individual_day_order.sql}

\subsubsection{Dodawanie danych zamawiającego}
\lstinputlisting{sql/procedures/add_buyer.sql}

\subsubsection{Dodawanie danych uczestnika}
\lstinputlisting{sql/procedures/add_attendee_day.sql}

\subsubsection{Dodawanie wykładu}
\lstinputlisting{sql/procedures/add_lecture.sql}

\subsubsection{Dodawanie warsztatu}
\lstinputlisting{sql/procedures/add_workshop.sql}

\subsubsection{Dodawanie ceny dnia konferencji}
\lstinputlisting{sql/procedures/add_price.sql}

\subsubsection{Dodawanie płatności}
\lstinputlisting{sql/procedures/add_payment.sql}

\subsection{Aktualizujące}
\subsubsection{Unieważnienie nieopłaconych zamówień}
Procedura oznacza zamówienia nieopłacone w przeciagu 7 dni od złożenia jako anulowane.
Powinna być uruchamiana raz dziennie, np. wywołaniem z Harmonogramu Zadań.

\lstinputlisting{sql/procedures/update_overdue_orders.sql}

\subsubsection{Zmiana limitu miejsc w warsztacie}
Procedura sprawdza, czy liczba już zarezerwowwanych miejsc nie przekracza nowego limitu i zmienia limit miejsc na warsztacie.
\lstinputlisting{sql/procedures/update_workshop_attendees_limit.sql}

\subsubsection{Zmiana limitu miejsc w dniu konferencji}
Procedura sprawdza, czy liczba już zarezerwowwanych miejsc nie przekracza nowego limitu i zmienia limit miejsc w dniu konferencji. Jeśli któryś z warsztatów odbywających się tego dnia ma wyższy limit również jest on zmieniany.
\lstinputlisting{sql/procedures/update_workshop_attendees_limit.sql}



\section{Triggery}


\subsection{Propagacja anulowania zamówienia}
Trigger propagujący ustawienie pola \texttt{is\_canceled} z tabeli \texttt{order} do tabel \texttt{order\_item\_day} i \texttt{order\_item\_workshop}.

\lstinputlisting{sql/triggers/order_canceled_trigger.sql}

\subsection{Sprawdenie wystarczajacej liczby miejsc w dniu konferencji}
\lstinputlisting{sql/triggers/too_few_places_conference_day_trigger.sql}

\subsection{Sprawdenie wystarczajacej liczby miejsc na warsztacie}
\lstinputlisting{sql/triggers/too_few_places_workshop_trigger.sql}

\subsection{Sprawdzenie limitu przypisania uczestników do rezerwacji}
Trigger sprawdza, czy liczba uczestników (\textit{attendees}) przypisanych do zamówienia nie przekracza wielkości rezerwacji.

\lstinputlisting{sql/triggers/attendees_not_above_reservation_trigger.sql}

\subsection{Sprawdzenie limitu przypisania uczestników warsztatu do rezerwacji}

\lstinputlisting{sql/triggers/workshop_attendees_not_above_reservation_trigger.sql}

\subsection{Sprawdzenie zapowiedzianej liczby studentów}
Trigger sprawdza, czy po uzupełnieniu danych uczestników liczba studentów jest zgodna z rezeracją.
\lstinputlisting{sql/triggers/students_in_day_order_match_attendees.sql}

\subsection{Sprawdzenie zapowiedzianej liczby studentów}
Trigger sprawdza, czy po uzupełnieniu danych uczestników liczba studentów biorących udział w warsztacie jest zgodna z rezeracją.
\lstinputlisting{sql/triggers/students_in_workshop_order_match_attendees.sql}

\subsection{Sprawdenie limitów dnia konferencji i warsztatu }
Trigger sprawdza, czy limit uczestników na dzien konferencji jest większy lub równy niż na warsztat.
\lstinputlisting{sql/triggers/conference_day_limit_over_workshops_limit_trigger.sql}

\subsection{Sprawdenie liczby uczestników warsztatu i dnia konferencji dla zmówienia}
Trigger sprawdza, czy liczba zarejestrowanych uczestników na warsztaty nie przekracza zarejstrowanych na dzień konferencji dla zamówienia.
\lstinputlisting{sql/triggers/day_attendees_more_than_workshop_attendees_trigger.sql}

\subsection{Sprawdenie liczby rezerwowanyhc konferenji w jednym zamówieniu}
\lstinputlisting{sql/triggers/one_conference_per_order_trigger.sql}

\subsection{Sprawdenie przekroczenia 7 dni na zapłatę od daty zamówienia}
\lstinputlisting{sql/triggers/payment_seven_days_after_order_trigger.sql}

\subsection{Sprawdenie czy klient indywidualny podaje dane uczestników z zamówieniem.}
Zamówienie składane przez klienta powinno być realizowane procedurą \texttt{add\_indivudal\_day\_order} która od razu uzupełnia jego dane jako uczestnika.
\lstinputlisting{sql/triggers/private_buyer_must_provide_attendee_trigger.sql}

\subsection{Sprawdenie nienachodzenia czasów warsztatów uczestników}
\lstinputlisting{sql/triggers/attendee_only_one_workshop_at_one_time_trigger.sql}

\subsection{Sprawdenie nieprzekraczania przez termin ceny terminu konferencji}

\lstinputlisting{sql/triggers/price_end_not_after_conference_day_trigger.sql}


\subsection{Sprawdenie istnienia ceny do początku dnia konferencji}

\lstinputlisting{sql/triggers/exists_price_till_beginning_of_conference_day_trigger.sql}

\subsection{Sprawdenie liczby rezerwowanych miejsc przez klienta indywidualnego}
\lstinputlisting{sql/triggers/one_attendee_per_private_buyer_trigger.sql}

\subsection{Sprawdenie uprzedniości daty zamówienia do daty dnia konferencji}
\lstinputlisting{sql/triggers/order_before_conference_day_trigger.sql}

\subsection{Sprawdenie liczby uczestników warsztatów i dnia koferencji dla zamówienia}
Trigger sprawdzajacy, czy liczby uczestników warsztatów danego zamówienia jest nie wieksza od liczby uczestników dnia konferencji, do którego przypisany jest warsztat.
\lstinputlisting{sql/triggers/workshop_attendees_fewer_than_day_attendees_trigger.sql}

\subsection{Sprawdenie limitów dnia konferencji i warsztatu}
Trigger sprawdza, czy limit dnia konferencji jest wiekszy od warsztatu bedacego w relacji z tym dniem.
\lstinputlisting{sql/triggers/workshop_limit_below_day_limit_trigger.sql}

\section{Indeksy}
Oprócz indeksów tworzonych automatycznie w związku z poleceniami \texttt{CONSTRAINT PRIMARY KEY} i \texttt{CONSTRAINT UNIQUE} stworzone zostały indeksy majace na celu przyspieszenie wyszukiwania równoczesnych warsztatów uczestnika:

\lstinputlisting{sql/indexes.sql}

\section{Proponowane role w systemie}
Role przedstawiane są w kolejności rosnących uprawnień, to znaczy role następne mają również uprawnienai dostępne wcześniejszym.

\subsection{System www}
Tymi uprawnieniami posługiwać się będzie witryna sieci web służąca klientom do przeglądania oferty firmy.

Ma dostęp do odczytu informacji o przyszłych konferencjach (widok nadchodzących konferencji, procedura podajaca wydarzenia danego dnia). Może wywoływać procedury dodajace rezerwację.

\subsection{Osoba kontaktujaca się z klientami}
Ma dostęp do list uczestników danej konferencji lub warszatatu. 

\subsection{Osoba obsługujaca klientów biznesowych}
Ma dostęp do do widoku brakujących danych klientów, może dodawać zamówienia klientów oraz uzupełniać dane uczestników.

\subsection{Koordynator konferencji}
Może dodawać i modyfikować konferencje, warsztaty, wykłady i progi cenowe.

\subsection{Zarząd firmy}
Ma dostęp do odczytu wszystkich tabel i widoków, w szczególności widoków i procedur ze statystykami finansowymi.

\subsection{Administrator}
Pełen dostęp do przeglądania i modyfikowania bazy danych.


\section{Generator danych}

Do napisania generatora danych posłużyliśmy się językiem \textbf{Elixir}. Wybór podyktowany był zaciekawieniem tym młodym językiem funkcyjnym.

Do komunikacji z bazą danych wykorzystana została biblioteka \textbf{Ecto} ( \url{https://github.com/elixir-ecto/ecto} ), a do generowania losowych wartości biblioteka \textbf{Faker} (\url{https://github.com/igas/faker}).


\input{generator.tex}

\end{document}
